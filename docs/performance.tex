\chapter{Performance Evaluation}

\section{Testing Methodologies}

The performance of our tool was evaluated using a combination of manual testing and automated testing methodologies. The testing process included:

\begin{itemize}
    \item \textbf{Unit Testing:} Individual components and modules were tested in isolation to ensure they function correctly.
    \item \textbf{Integration Testing:} The interaction between different modules was tested to verify that they work together as expected.
    \item \textbf{System Testing:} The entire system was tested in real-world scenarios to assess its overall performance and reliability.
    \item \textbf{Load Testing:} The system was subjected to varying levels of load to evaluate its performance under heavy usage.
\end{itemize}

\section{Performance Metrics and Benchmarks}

Performance metrics and benchmarks were defined to measure various aspects of the system's performance. Key metrics include:

\begin{itemize}
    \item \textbf{Simulation Speed:} The time taken to complete a simulation, measured in seconds.
    \item \textbf{Rendering Frame Rate:} The number of frames rendered per second during simulation visualization.
    \item \textbf{Memory Usage:} The amount of memory consumed by the application during simulation.
    \item \textbf{CPU Utilization:} The percentage of CPU resources utilized by the application.
\end{itemize}

Benchmarks were established based on these metrics to evaluate the performance of the system under different conditions.

\section{Results of Performance Tests}

The performance tests were conducted under various scenarios and conditions to evaluate the system's performance comprehensively. The results of the performance tests are summarized below:

\begin{itemize}
    \item \textbf{Simulation Speed:} The average simulation speed was found to be approximately 10 simulations per second, varying depending on the complexity of the simulation and the hardware specifications.
    \item \textbf{Rendering Frame Rate:} The rendering frame rate remained stable at around 60 frames per second for most simulations, providing smooth and responsive visualization.
    \item \textbf{Memory Usage:} The memory usage of the application was moderate, typically ranging from 100 MB to 500 MB depending on the size of the simulation and the number of objects.
    \item \textbf{CPU Utilization:} The CPU utilization was found to be around 30% to 50% during simulations, indicating efficient resource utilization.
\end{itemize}

Overall, the performance tests demonstrated that the system is capable of handling simulations efficiently and providing a satisfactory user experience.

\section{Comparison with Other Tools (if Applicable)}

In comparison with existing modeling and simulation tools, our tool offers several advantages, including:

\begin{itemize}
    \item \textbf{Ease of Use:} Our tool provides a user-friendly interface and intuitive controls, making it accessible to users of all skill levels.
    \item \textbf{Real-time Visualization:} The real-time 3D visualization capabilities of our tool offer a more immersive and interactive simulation experience compared to traditional tools.
    \item \textbf{Web-Based:} Being web-based, our tool eliminates the need for installation and allows for easy access from any device with a web browser.
    \item \textbf{Performance:} The performance tests have shown that our tool is capable of handling simulations efficiently, with stable frame rates and moderate resource usage.
\end{itemize}

While existing tools such as MATLAB offer advanced features and capabilities, they are often complex and require specialized knowledge to use effectively. Our tool aims to bridge this gap by providing a simple yet powerful platform for modeling and simulation.
