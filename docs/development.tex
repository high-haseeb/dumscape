\chapter{Development Process}

\section{Coding Techniques and Algorithms Used}

The development of our project involved several coding techniques and algorithms to ensure efficient and accurate simulations. Key techniques and algorithms used include:

\subsection{Object-Oriented Programming (OOP)}

OOP was extensively used to model physical objects and their interactions. Each object in the simulation, such as particles, rigid bodies, and forces, was represented as a class with properties and methods.

\subsection{Numerical Integration}

Numerical integration methods, such as the Euler method and the Runge-Kutta methods, were implemented to solve differential equations. These methods approximate the solution by iteratively updating the state of the system over small time steps.

\begin{lstlisting}[language=JavaScript, caption=Euler Integration Method]
function integrateEuler(object, dt) {
    // Update position
    object.position.x += object.velocity.x * dt;
    object.position.y += object.velocity.y * dt;
    object.position.z += object.velocity.z * dt;
    
    // Update velocity
    object.velocity.x += object.acceleration.x * dt;
    object.velocity.y += object.acceleration.y * dt;
    object.velocity.z += object.acceleration.z * dt;
}
\end{lstlisting}

\subsection{Collision Detection and Response}

Efficient algorithms for collision detection and response were crucial for realistic simulations. The Separating Axis Theorem (SAT) and Bounding Volume Hierarchies (BVH) were used to detect and resolve collisions.

\begin{lstlisting}[language=JavaScript, caption=Collision Detection Using SAT]
function detectCollisionSAT(objectA, objectB) {
    // Calculate projections of objects onto potential separating axes
    // Check for overlap in all axes
    // If overlap exists in all axes, collision is detected
    // Calculate response forces and update velocities accordingly
}
\end{lstlisting}

\section{Challenges Faced and Solutions Implemented}

During the development of the project, several challenges were encountered. Key challenges and their solutions include:

\subsection{Performance Optimization}

Simulating complex physical systems in real-time required significant computational power. To address this, we optimized the Physics Engine by implementing spatial partitioning techniques, such as Quadtrees and Octrees, to reduce the number of collision checks.

\begin{lstlisting}[language=JavaScript, caption=Spatial Partitioning with Octree]
function Octree(boundary, capacity) {
    this.boundary = boundary;
    this.capacity = capacity;
    this.objects = [];
    this.divided = false;
}

Octree.prototype.insert = function(object) {
    // Insert object into the octree
    // Subdivide if capacity is exceeded
};
\end{lstlisting}

\subsection{Precision and Stability}

Numerical errors and instability can accumulate over time in simulations. To mitigate this, we used higher-order integration methods, such as the Runge-Kutta 4th order method, and implemented techniques to correct small errors periodically.

\subsection{User Interface Design}

Creating an intuitive and responsive UI that allows users to interact with the simulation in real-time was challenging. We employed modern web development frameworks, such as React.js, to build a dynamic and responsive interface.

\section{Details of the Custom Physics Engine}

The custom Physics Engine is designed to handle various physical phenomena and interactions. Key components of the Physics Engine include:

\subsection{Force Calculation}

The Physics Engine calculates forces acting on each object based on user-defined parameters and physical laws. This includes gravitational forces, frictional forces, and applied forces.

\begin{lstlisting}[language=JavaScript, caption=Force Calculation]
function calculateForces(objects) {
    objects.forEach(object => {
        // Reset net force
        object.force.set(0, 0, 0);
        
        // Apply gravitational force
        object.force.y -= object.mass * 9.81;
        
        // Apply frictional force
        object.force.x -= object.velocity.x * object.friction;
        object.force.z -= object.velocity.z * object.friction;
    });
}
\end{lstlisting}

\subsection{Integration and State Update}

The engine integrates the equations of motion to update the states of objects. This involves updating positions, velocities, and accelerations based on the calculated forces.

\begin{lstlisting}[language=JavaScript, caption=State Update Using Runge-Kutta Method]
function integrateRungeKutta(object, dt) {
    // Implement the Runge-Kutta 4th order integration method
    // Update object state
}
\end{lstlisting}

\subsection{Collision Detection and Response}

The engine detects and responds to collisions between objects, ensuring conservation of momentum and energy. The response includes calculating impulse forces and updating velocities accordingly.

\begin{lstlisting}[language=JavaScript, caption=Collision Response]
function resolveCollision(objectA, objectB) {
    // Calculate relative velocity
    // Calculate impulse force
    // Update velocities of both objects
}
\end{lstlisting}

\section{Integration of the 3D Rendering Module}

The 3D Rendering module is integrated with the Physics Engine to visualize simulation results in real-time. This involves passing the updated states from the Physics Engine to the rendering module and using graphics techniques to display the objects.

\subsection{Rendering Pipeline}

The rendering pipeline includes setting up the scene, camera, and lighting, and rendering the objects based on their states. We used Three.js, a popular JavaScript library for 3D graphics, to implement the rendering pipeline.

\begin{lstlisting}[language=JavaScript, caption=Setting Up the 3D Scene with Three.js]
function setupScene() {
    var scene = new THREE.Scene();
    var camera = new THREE.PerspectiveCamera(75, window.innerWidth / window.innerHeight, 0.1, 1000);
    var renderer = new THREE.WebGLRenderer();
    
    renderer.setSize(window.innerWidth, window.innerHeight);
    document.body.appendChild(renderer.domElement);
    
    var geometry = new THREE.BoxGeometry();
    var material = new THREE.MeshBasicMaterial({ color: 0x00ff00 });
    var cube = new THREE.Mesh(geometry, material);
    scene.add(cube);
    
    camera.position.z = 5;
    
    var animate = function () {
        requestAnimationFrame(animate);
        
        // Update object states from Physics Engine
        
        renderer.render(scene, camera);
    };
    
    animate();
}
\end{lstlisting}

\subsection{Updating Objects for Rendering}

The Physics Engine provides the updated states of objects, which are then used by the rendering module to display the objects in their new positions and orientations.

\begin{lstlisting}[language=JavaScript, caption=Updating Object States for Rendering]
function updateRendering(objects) {
    objects.forEach(object => {
        // Update 3D object position and rotation based on physics state
        object.mesh.position.set(object.position.x, object.position.y, object.position.z);
        object.mesh.rotation.set(object.rotation.x, object.rotation.y, object.rotation.z);
    });
}
\end{lstlisting}

By integrating the Physics Engine and the 3D Rendering module, our project provides a dynamic and interactive simulation environment that allows users to visualize and interact with physical phenomena in real-time.
