\chapter{Introduction}

During my Modeling \& Simulation class, I often found myself puzzled by the abstract nature of the subject. Understanding these concepts through equations and graphs alone felt inadequate, and I began to search for a more effective way to visualize them. My background in game development provided me with a practical perspective on simulation. In game development, we simulate various aspects of reality, such as physics, lighting, and interactions, which helped me understand the essence of simulation in a tangible context.
\\

  This realization led me to the idea of creating a tool that could bridge the gap between theoretical modeling and practical visualization. The aim was to develop a tool that is not only easily accessible to all but also user-friendly and capable of encapsulating the numerical aspects of modeling and simulation. Such a tool would offer a straightforward understanding of the subject by connecting theoretical concepts with real-world scenarios. By allowing users to experiment with various scenarios, which they have previously encountered only in the form of equations and transfer functions, the tool would make learning more interactive and engaging.
\\

  One of the key challenges in simulation is that computers operate in discrete steps, limited by the clock frequency of their processors. While simulation is inherently a continuous process, we can approximate it by calculating the model at a sufficiently high frequency. This approach can provide a rough estimate that is close to continuous behavior, making it possible to visualize complex phenomena in a comprehensible manner.
\\

  In designing this tool, I focused on several key features to enhance its usability and educational value. First, the tool should provide intuitive controls that allow users to manipulate parameters and immediately see the effects on the simulation. This immediate feedback loop is crucial for understanding the dynamic nature of systems being modeled. Second, it should offer a variety of pre-built scenarios that cover common topics in modeling and simulation, enabling users to explore these scenarios without needing to build models from scratch. Finally, the tool should be web based, and should not be too demanding in terms of resources.
\\

  By integrating these features, the tool aims to demystify the complex concepts of modeling and simulation, making them accessible and understandable to a wider audience. It can serve as a valuable educational resource, helping students and professionals alike to develop a deeper understanding of how theoretical models translate into real-world behaviors. Ultimately, this tool represents a step towards bridging the gap between abstract mathematical models and tangible, interactive experiences, fostering a more comprehensive understanding of the subject.
\\

  The tool is by no means complete and I aim to polish it and make it better.

