\chapter{Literature Review}

\section{Overview of Existing Modeling and Simulation Tools}

Modeling and simulation are essential tools in various fields such as engineering, physics, and computer science. There are several established tools available for these purposes, each with its own set of features and limitations. Some of the most commonly used tools include:

\begin{itemize}
    \item \textbf{MATLAB:} A high-level language and interactive environment used by millions of engineers and scientists worldwide. It is renowned for its numerical computing capabilities and extensive libraries for data analysis, visualization, and algorithm development.
    \item \textbf{Simulink:} An add-on product to MATLAB, which provides a graphical environment for simulation and Model-Based Design of multi-domain dynamic and embedded systems.
    \item \textbf{COMSOL Multiphysics:} A simulation software for modeling designs, devices, and processes in all fields of engineering, manufacturing, and scientific research.
    \item \textbf{ANSYS:} A comprehensive software suite that spans the entire range of physics, providing access to virtually any field of engineering simulation that a design process requires.
\end{itemize}

\section{Comparison of Features in Existing Tools}

While these tools are powerful, they have certain limitations when it comes to ease of use, interactivity, and web accessibility. Below is a comparison of some key features:

\begin{tabular}{|l|c|c|c|c|}
    \hline
    \textbf{Feature} & \textbf{MATLAB} & \textbf{Simulink} & \textbf{COMSOL} & \textbf{ANSYS} \\
    \hline
    Numerical Computing & Excellent & Good & Excellent & Excellent \\
    \hline
    GUI & Yes & Yes & Yes & Yes \\
    \hline
    3D Visualization & Limited & Limited & Good & Excellent \\
    \hline
    Web-Based Access & No & No & Limited & Limited \\
    \hline
    Interactivity & Limited & Limited & Good & Good \\
    \hline
    Ease of Use & Moderate & Moderate & Complex & Complex \\
    \hline
\end{tabular}

\section{Gaps in Current Tools}

Despite the strengths of these existing tools, there are notable gaps that our web-based app aims to fill:

\begin{itemize}
    \item \textbf{Interactive 3D UI:} Current tools like MATLAB and Simulink offer limited capabilities for interactive 3D visualization. Our app provides a more dynamic and interactive 3D user interface, enhancing the user's ability to visualize and interact with the simulation in real time.
    \item \textbf{Web-Based Accessibility:} Existing tools are typically desktop-based, which can limit accessibility. Our app is web-based, allowing users to access it from anywhere with an internet connection, thus promoting ease of use and collaboration.
    \item \textbf{Speed and Efficiency:} MATLAB, while powerful, can be slow for large-scale simulations due to its interpreted nature. Our app, with a custom-built physics engine, is optimized for speed and performance, offering faster simulation times.
    \item \textbf{User-Friendly Interface:} Many existing tools have a steep learning curve and can be convoluted. Our app focuses on providing a user-friendly interface that simplifies the process of setting up and running simulations, making it accessible even to those with limited technical background.
\end{itemize}

In summary, our app addresses significant gaps in the current modeling and simulation tools by offering an interactive, web-based, and efficient solution with a focus on user-friendliness and real-time 3D visualization.



